\documentclass{beamer}

\usepackage[utf8]{inputenc}
\usepackage{listings}

\title[Template functions] {Template functions in C++ and generic programming}
\author[Vladimirov]{K.~Vladimirov}

\begin{document}
\frame{\titlepage}

\begin{frame}[fragile]
\frametitle{Simple function templates}

\begin{columns}[c]
\begin{column}{.5\textwidth}
\begin{block}{Simple template function}
\begin{lstlisting}[basicstyle=\tiny]
template <typename T> inline const T
min (const T a, const T b)
{
  return (a < b) ? a : b;
}
\end{lstlisting}
\end{block}
\end{column}
\begin{column}{.5\textwidth}
\begin{block}{Syntax, according to (14.1) C++14}
\small{
tdecl: template $\langle$ plist $\rangle$ declaration\\
plist: parameter $\mid$ (plist , parameter)
}
\end{block}
\end{column}
\end{columns}\pause

\begin{alertblock}{What about case, where type \lstinline!T! denotes heavy argument?}
In this case, say \lstinline!sizeof(T)>1000!, passing on stack is overhead
\end{alertblock}\pause

\begin{block}{Use constant references to avoid overhead}
\begin{lstlisting}[basicstyle=\small]
template <typename T> inline const T&
min (const T &a, const T &b)
{
  return (a < b) ? a : b;
}
\end{lstlisting}
\end{block}

\end{frame}

\begin{frame}[fragile]
\frametitle{How to write and test min/max pair?}

\begin{columns}[c]
\begin{column}{.5\textwidth}
\begin{block}{Min function}
\begin{lstlisting}[basicstyle=\tiny]
template <typename T> inline const T&
min (const T &a, const T &b)
{
  return (a < b) ? a : b;
}
\end{lstlisting}
\end{block}
\end{column}
\begin{column}{.5\textwidth}
\begin{block}{Max function}
\begin{lstlisting}[basicstyle=\tiny]
template <typename T> inline const T&
max (const T &a, const T &b)
{
  return (a > b) ? a : b;
}
\end{lstlisting}
\end{block}
\end{column}
\end{columns}\pause

\begin{alertblock}{How to test this solution?}
Idea: assert, that $(x<y) \Rightarrow \{\min(x,y),\max(x,y)\}=\{x,y\}$
\end{alertblock}\pause

\begin{block}{Write template unit tests}
\begin{lstlisting}[basicstyle=\tiny]
template <typename T> bool
test_minmax (const T &a, const T &b)
{
  assert (a <= b);
  return (min (a, b) == a) && (max (a, b) == b);
}
\end{lstlisting}

We can call this function with different data:
\begin{lstlisting}[basicstyle=\tiny]
assert(test_minmax ('a', 'b'));
assert(test_minmax (5, 6));
assert(test_minmax (3.0, 7.2));
/* ... */
\end{lstlisting}
\end{block}

\end{frame}

\begin{frame}[fragile]
\frametitle{More interesting tests and unexpected bugs}

\begin{alertblock}{What if \lstinline!T! is more sophisticated?}
Say we do have class Person with name and age, and comparison is based on age:
\begin{lstlisting}[basicstyle=\tiny]
struct Person
{
  const char *name;
  int age;
  Person (const char *a_name, int an_age) : name (a_name), age(an_age) {}
};

/* operators > and <= are the same */
static bool
operator < (const Person &lhs, const Person &rhs)
{
  return lhs.age < rhs.age;
}

static bool
operator == (const Person &lhs, const Person &rhs)
{
  return (lhs.age == rhs.age) && (!strcmp (lhs.name, rhs.name));
}
\end{lstlisting}

What if we will call 

\begin{lstlisting}[basicstyle=\tiny]
Person Ivan ("Ivan", 20);
Person Danila ("Danila", 20);
assert(test_minmax (3.0, 7.2));
\end{lstlisting}
\end{alertblock}\pause
\end{frame}

\end{document}
